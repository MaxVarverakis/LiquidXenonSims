% ****** Start of file apssamp.tex ******
%
%   This file is part of the APS files in the REVTeX 4.2 distribution.
%   Version 4.2a of REVTeX, December 2014
%
%   Copyright (c) 2014 The American Physical Society.
%
%   See the REVTeX 4 README file for restrictions and more information.
%
% TeX'ing this file requires that you have AMS-LaTeX 2.0 installed
% as well as the rest of the prerequisites for REVTeX 4.2
%
% See the REVTeX 4 README file
% It also requires running BibTeX. The commands are as follows:
%
%  1)  latex apssamp.tex
%  2)  bibtex apssamp
%  3)  latex apssamp.tex
%  4)  latex apssamp.tex
%
\documentclass[%
reprint,
%superscriptaddress,
%groupedaddress,
%unsortedaddress,
%runinaddress,
%frontmatterverbose, 
%preprint,
%preprintnumbers,
%nofootinbib,
%nobibnotes,
%bibnotes,
amsmath, amssymb,
aps,
%pra,
%prb,
%rmp,
%prstab,
%prstper,
%floatfix,
]{revtex4-2}

\usepackage{graphicx}% Include figure files
\usepackage{dcolumn}% Align table columns on decimal point
\usepackage{bm}% bold math
%\usepackage{hyperref}% add hypertext capabilities
%\usepackage[mathlines]{lineno}% Enable numbering of text and display math
%\linenumbers\relax % Commence numbering lines

%\usepackage[showframe,%Uncomment any one of the following lines to test 
%%scale=0.7, marginratio={1:1, 2:3}, ignoreall,% default settings
%%text={7in,10in},centering,
%%margin=1.5in,
%%total={6.5in,8.75in}, top=1.2in, left=0.9in, includefoot,
%%height=10in,a5paper,hmargin={3cm,0.8in},
%]{geometry}

\begin{document}

\preprint{APS/123-QED}

\title{Using a Liquid Xenon Positron Target}% Force line breaks with \\
% \thanks{A footnote to the article title}%

\author{Max Varverakis}
\email{mvarvera@calpoly.edu}
\affiliation{California Polytechnic State University, San Luis Obispo, CA 93407, USA}
\author{Spencer Gessner}%
\email{sgess@slac.stanford.edu}
\affiliation{SLAC National Accelerator Laboratory, Menlo Park, California 94025, USA}%

% \collaboration{MUSO Collaboration}%\noaffiliation

% \author{Charlie Author}
%  \homepage{http://www.Second.institution.edu/~Charlie.Author}
% \affiliation{
%  Second institution and/or address\\
%  This line break forced% with \\
% }%
% \affiliation{
%  Third institution, the second for Charlie Author
% }%
% \author{Delta Author}
% \affiliation{%
%  Authors' institution and/or address\\
%  This line break forced with \textbackslash\textbackslash
% }%

% \collaboration{CLEO Collaboration}%\noaffiliation

\date{\today}

\begin{abstract}
ABSTRACT
\begin{description}
\item[Usage]
Secondary publications and information retrieval purposes.
\item[Structure]
You may use the \texttt{description} environment to structure your abstract;
use the optional argument of the \verb+\item+ command to give the category of each item. 
\end{description}
\end{abstract}

%\keywords{Suggested keywords}%Use showkeys class option if keyword display desired
\maketitle

%\tableofcontents

\section{Outline}
\begin{itemize}
    \item Introduction
    \begin{itemize}
        \item What is the problem that we are trying to solve?
        \item What are the issues with "traditional" positron targets?
        \item What approaches have been tried already?
        \item How many positrons-per-second are needed for Linear Collider applications?
        \item Introduction is basically a literature survey/review.
    \end{itemize}
    \item Comparing Positron production in Xenon vs W or Ta
    \begin{itemize}
        \item This is where the GEANT simulations go.
        \item How thick/dense does Xenon need to be to match positron production in W or Ta?
    \end{itemize}
    \item Cryo-cooled Xenon gas jets
    \begin{itemize}
        \item Does this exist?
        \item Describe work with liquid Xenon and work with cryo-cooled gas jets at SLAC.
        \item Vacuum challenges?
    \end{itemize}
    \item Conclusion
    \begin{itemize}
        \item Describe next steps. How would we actually build/implement this?
    \end{itemize}
    
\end{itemize}

\section{Introduction}
A common scheme for producing positrons is by colliding high energy electrons into a high-Z target.
The collision between an electron beam and a solid target generates an electromagnetic particle shower,
in which positrons are produced.
Because the collision is such high energy, a great deal of energy is deposited in the target in the form of
thermal energy.  As a result, solid targets tend to degrade over time [].  Since positron yield increases as a
function of radiation length [], a thicker the target implies a greater positron yield, but that also implies
a greater energy will deposited into the target, leading to a quicker degredation of the target.

There are various methods for increasing
the life span of solid targets, such as using a cooling system [] and rotating the target so that the beam doesn't
hit the same spot of the target every pulse [].

Previous experiments have been carried out to explore alternatives to using solid targets, such as using liquid Mercury (Hg),
but the apparent hazards that Hg presents are too dangerous to implement in any efficient manner.
Other approaches include...

For typical Linear Collider applications, around *** $e^+$ per second need to be produced [].

In this paper, we explore the possibility of using a liquid Xenon (Xe) target to produce positrons.

\section{Simulation Results}
Comparison study between Tantalum (Ta) and liquid Xe because we have a reference study on Ta [].
We used GEANT4 to simulate the collision between 10 GeV $e^-$ and a target.  We compare the results of using a
Ta target and a liquid Xe target.

See Table \ref{tab:G4Params} for parameters used in the simulation.

\begin{table}[h]
    \centering
    \begin{tabular}{||c|c|c|c||}
        \hline
        Material & Z & Density [$\textrm{g} \cdot \textrm{cm}^{-3} $] & Radiation Length [cm] \\
        \hline \hline
        Tantalum (Ta) & 73 & 16.654 & 0.4094 \\
        \hline
        Liquid Xe (Xe) & 54 & 2.953 & 2.872 \\
        \hline
    \end{tabular}
    \caption{Parameters used in GEANT4 simulation.}
    \label{tab:G4Params}
\end{table}

% \begin{figure}
%     \includegraphics{../images/CompDeps.png}
%     \caption{\label{fig:EDep}Energy deposition in Ta and liquid Xe targets per incident electron at 10 GeV.}
% \end{figure}

% \begin{figure}
%     \includegraphics{../images/}
%     \caption{\label{fig:EDep}Energy deposition in Ta and liquid Xe targets per incident electron at 10 GeV.}
% \end{figure}

As seen in Figure *ref*, the max positron yield for both Ta and liquid Xe occurs at around 2.75 radiation lengths.

\section{Cryo-cooled Xenon Gas Jets/Liquid Xenon}

\section{Conclusion}
Source code for GEANT4 simulations can be found at https://github.com/MaxVarverakis/LiquidXenonSims.git.

\pagebreak
\begin{acknowledgments}
We wish to acknowledge the support of the author community in using
REV\TeX{}, offering suggestions and encouragement, testing new versions,
\dots.
\end{acknowledgments}

\appendix

\section{Appendixes}

To start the appendixes, use the \verb+\appendix+ command.
This signals that all following section commands refer to appendixes
instead of regular sections. Therefore, the \verb+\appendix+ command
should be used only once---to setup the section commands to act as
appendixes. Thereafter normal section commands are used. The heading
for a section can be left empty. For example,
\begin{verbatim}
\appendix
\section{}
\end{verbatim}
will produce an appendix heading that says ``APPENDIX A'' and
\begin{verbatim}
\appendix
\section{Background}
\end{verbatim}
will produce an appendix heading that says ``APPENDIX A: BACKGROUND''
(note that the colon is set automatically).

If there is only one appendix, then the letter ``A'' should not
appear. This is suppressed by using the star version of the appendix
command (\verb+\appendix*+ in the place of \verb+\appendix+).

\section{A little more on appendixes}

Observe that this appendix was started by using
\begin{verbatim}
\section{A little more on appendixes}
\end{verbatim}

Note the equation number in an appendix:
\begin{equation}
E=mc^2.
\end{equation}

\subsection{\label{app:subsec}A subsection in an appendix}

You can use a subsection or subsubsection in an appendix. Note the
numbering: we are now in Appendix~\ref{app:subsec}.

Note the equation numbers in this appendix, produced with the
subequations environment:
\begin{subequations}
\begin{eqnarray}
E&=&mc, \label{appa}
\\
E&=&mc^2, \label{appb}
\\
E&\agt& mc^3. \label{appc}
\end{eqnarray}
\end{subequations}
They turn out to be Eqs.~(\ref{appa}), (\ref{appb}), and (\ref{appc}).

% The \nocite command causes all entries in a bibliography to be printed out
% whether or not they are actually referenced in the text. This is appropriate
% for the sample file to show the different styles of references, but authors
% most likely will not want to use it.
\nocite{*}

\bibliography{apssamp}% Produces the bibliography via BibTeX.

\end{document}
%
% ****** End of file apssamp.tex ******
